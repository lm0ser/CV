%!TEX TS-program = xelatex
\documentclass[]{friggeri-cv}
\usepackage{afterpage}
\usepackage{hyperref}
\usepackage{color}
\usepackage{xcolor}
\usepackage{smartdiagram}
\usepackage{fontspec}
% if you want to add fontawesome package
% you need to compile the tex file with LuaLaTeX
% References:
%   http://texdoc.net/texmf-dist/doc/latex/fontawesome/fontawesome.pdf
%   https://www.ctan.org/tex-archive/fonts/fontawesome?lang=en
%\usepackage{fontawesome}
\usepackage{metalogo}
\usepackage{dtklogos}
\usepackage[utf8]{inputenc}
\usepackage{tikz}
\usetikzlibrary{mindmap,shadows}
\hypersetup{
    pdftitle={},
    pdfauthor={},
    pdfsubject={},
    pdfkeywords={},
    colorlinks=false,           % no lik border color
    allbordercolors=white       % white border color for all
}
\smartdiagramset{
    bubble center node font = \footnotesize,
    bubble node font = \footnotesize,
    % specifies the minimum size of the bubble center node
    bubble center node size = 0.5cm,
    %  specifies the minimum size of the bubbles
    bubble node size = 0.5cm,
    % specifies which is the distance among the bubble center node and the other bubbles
    distance center/other bubbles = 0.3cm,
    % sets the distance from the text to the border of the bubble center node
    distance text center bubble = 0.5cm,
    % set center bubble color
    bubble center node color = pblue,
    % define the list of colors usable in the diagram
    set color list = {lightgray, materialcyan, orange, green, materialorange, materialteal, materialamber, materialindigo, materialgreen, materiallime},
    % sets the opacity at which the bubbles are shown
    bubble fill opacity = 0.6,
    % sets the opacity at which the bubble text is shown
    bubble text opacity = 0.5,
}

\addbibresource{bibliography.bib}
\RequirePackage{xcolor}
\definecolor{pblue}{HTML}{0395DE}

\begin{document}
\header{Lucas}{MOSER}
      {Administrateur systèmes et réseaux}
      
% Fake text to add separator      
\fcolorbox{white}{gray}{\parbox{\dimexpr\textwidth-2\fboxsep-2\fboxrule}{%
.....
}}

% In the aside, each new line forces a line break
\begin{aside}
  \includegraphics[scale=0.15]{img/snow_circle.png}
  \section{Adresse}
    58 rue des pelicans
    12850 Onet-le-château
    ~
  \section{Téléphone}
    0678396842
    ~
  \section{Mail}
    \href{mailto:lucas.moser@live.fr}{\textbf{lucas.moser@}\\live.fr}
    ~
  \section{Web \& Git}
    \href{http://www.zone-58.com}{zone-58.com}
    \href{https://github.com/lm0ser}{github.com/lm0ser}
    ~
  % use  \hspace{} or \vspace{} to change bubble size, if needed
  \section{Compétences}
    \smartdiagram[bubble diagram]{
        \textbf{C/C++},
        \textbf{Python},
        \textbf{Java},
        \textbf{Lua/UCI},
        \textbf{Other\vspace{3mm}},
        \textbf{HTML/CSS}\\\textbf{JS/jQuery},
        \textbf{PHP},
        \textbf{Android}
    }
    ~
  \section{Personel}
    \smartdiagram[bubble diagram]{
        \textbf{VTT},
        \textbf{Photos}\\\textbf{Vidéos},
        \textbf{Randonnée},
        \textbf{Drone}\\\textbf{Solving},
        \textbf{\vspace{2mm}Manage\vspace{2mm}}
    }
    ~
\end{aside}
~
\section{Expérience}
\begin{entrylist}
  \entry
    {Depuis 2003}
    {Administrateur systèmes et réseaux}
    {CCI de l'Aveyron (12)}
    {}
  \entry
    {01/02 - 06/03}
    {Web développeur }
    {Septime Création - Agence digitale (12)}
    {}
\end{entrylist}
\section{Diplômes}
\begin{entrylist}
  \entry
    {2001 - 2003}
    {CS2I - Concepteur de Systèmes d’Information}
    {CCI de l'Aveyron (12)}
    {}
  \entry
    {1999 - 2001}
    {BTS Informatique industrielle}
    {Lycée Jean Monnet à Aurillac (15)}
    {}
  \entry
    {1998}
    {BAC STI (génie électrotechnique)}
    {Lycée St Joseph à Rodez (12)}
    {}
\end{entrylist}
\section{Formations}
\begin{entrylist}
  \entry
    {2003}
    {Implémentation et gestion de Microsoft Exchange 2003 (MS2402)}
    {}
    {}
  \entry
    {2004}
    {}
    {}
    {}
  \entry
    {2005}
    {Linux administration (Niveau2)}
    {}
    {}
  \entry
    {2006}
    {Administration Windows Serveur 2003}
    {}
    {}
  \entry
    {2007}
    {Interconnecting Cisco Networking Devices (ICND1)}
    {}
    {}
  \entry
    {2008}
     {Interconnecting Cisco Networking Devices (ICND2)}
    {}
    {}
  \entry
    {2009}
    {A Vérifier}
    {}
    {}
  \entry
    {2010}
    {Mettre à jour ses compétences réseau et Active Directory vers Windows 2008 (6734A)}
    {}
    {}
  \entry
    {2011}
    {Administrer et gérer  VMware ESX 4.5}
    {}
    {}
    \entry
    {2012}
    {Configurer, gérer et maintenir Windows Server 2008 R2 (MS6419)}
    {}
    {}
   \entry
    {2013}
    {Configurer les services avancés de Windows Server 2012 R2 (M20412)}
    {}
    {}
  \entry
    {2015}
    {Implémentation de réseaux IP commutés Cisco (SWITCH)}
    {}
    {}
\end{entrylist}

\section{Projets}
\begin{entrylist}
  \entry
    {02/13 - Now}
    {Embedded Software Engineer}
    {Company Name}
    {Lorem ipsum dolor sit amet, consectetur adipiscing elit, sed do eiusmod tempor incididunt ut labore et dolore magna aliqua. Ut enim ad minim veniam, quis nostrud exercitation ullamco laboris nisi ut aliquip ex ea commodo consequat\\}
  \entry
    {01/12 - Now}
    {Co-Founder \& Developer}
    {Company Name}
    {Lorem ipsum dolor sit amet, consectetur adipiscing elit, sed do eiusmod tempor incididunt ut labore et dolore magna aliqua. Ut enim ad minim veniam, quis nostrud exercitation ullamco laboris nisi ut aliquip ex ea commodo consequat\\}
  \entry
    {06/10 - 09/10}
    {Part-time collaboration}
    {Company Name}
    {Lorem ipsum dolor sit amet, consectetur adipiscing elit, sed do eiusmod tempor incididunt ut labore et dolore magna aliqua. Ut enim ad minim veniam, quis nostrud exercitation ullamco laboris nisi ut aliquip ex ea commodo consequat\\}
  \entry
    {12/09 - 06/10}
    {Project Manager and Webmaster}
    {Company Name}
    {Lorem ipsum dolor sit amet, consectetur adipiscing elit, sed do eiusmod tempor incididunt ut labore et dolore magna aliqua. Ut enim ad minim veniam, quis nostrud exercitation ullamco laboris nisi ut aliquip ex ea commodo consequat\\}
\end{entrylist}
\newpage

\begin{aside}
~
~
~
  \section{OS Preference}
    \textbf{GNU/Linux}\includegraphics[scale=0.40]{img/5stars.png}
    \textbf{Unix}\includegraphics[scale=0.40]{img/4stars.png}
    \textbf{MacOS}\includegraphics[scale=0.40]{img/2stars.png}
    \textbf{Windows}\includegraphics[scale=0.40]{img/1stars.png}
    ~
  \section{Places Lived}
    \includegraphics[scale=0.25]{img/italia.png}
    ~
  \section{Languages}
    \textbf{Italian}\includegraphics[scale=0.40]{img/5stars.png}
    \textbf{English}\includegraphics[scale=0.40]{img/4stars.png}
    ~
\end{aside}

\section{Publications}
Author, Author, Author\\
\textbf{Lorem ipsum dolor sit amet, consectetur adipiscing elit, sed do eiusmod tempor incididunt ut labore et dolore magna aliqua}\\
\emph{Lorem ipsum dolor sit amet, consectetur adipiscing elit, sed do eiusmod tempor incididunt ut labore et dolore magna aliqua}
\\
\section{Honors \& Awards}
\begin{entrylist}
  \entry
    {10/2015}
    {Best swordsman duel}
    {Contest}
    {Lorem ipsum.\\
    \emph{Lorem ipsum}}
\end{entrylist}

\section{Certifications}
\begin{entrylist}
  \entry
    {02/2013}
    {Intro to Computer Science}
    {Udacity. E-learning}
    {\emph{Building a Python Search Engine}}
\end{entrylist}

\section{Other Info}
For the Italian job market:\\
\emph{Si autorizza il trattamento delle informazioni contenute nel curriculum in conformità alle disposizioni previste dal d.lgs. 196/2003. Si dichiara altresì di essere consapevole che, in caso di dichiarazioni non veritiere, si è passibili di sanzioni penali ai sensi del DPR 445/00 oltre alla revoca dei benefici eventualmente percepiti.}
\\
\begin{flushleft}
\emph{May 8th, 2016}
\end{flushleft}
\begin{flushright}
\emph{John Snow}
\end{flushright}

\end{document}
